%%%%%%%%%%%%%%%%%%%%%%%%%%%%%%%%%%%%%%%%%%%%%%%%%%%%%%%%%%%%%%%%%%%%%%%%
%% NOTE: If you find that it says                                     %%
%%                                                                    %%
%%                           1 of ??                                  %%
%%                                                                    %%
%% at the bottom of your first page, this means that the AUX file     %%
%% was not available when you ran LaTeX on this source. Simply RERUN  %% 
%% LaTeX to get the ``??'' replaced with the number of the last page  %% 
%% of the document. The AUX file will be generated on the first run   %%
%% of LaTeX and used on the second run to fill in all of the          %%
%% references.                                                        %%
%%%%%%%%%%%%%%%%%%%%%%%%%%%%%%%%%%%%%%%%%%%%%%%%%%%%%%%%%%%%%%%%%%%%%%%%

%%%%%%%%%%%%%%%%%%%%%%%%%%%% Document Setup %%%%%%%%%%%%%%%%%%%%%%%%%%%%

% Don't like 10pt? Try 11pt or 12pt
\documentclass[9pt]{article}

% This is a helpful package that puts math inside length specifications
\usepackage{calc}
\usepackage{amssymb}
\usepackage{multicol}


% Layout: Puts the section titles on left side of page
\reversemarginpar

%
%         PAPER SIZE, PAGE NUMBER, AND DOCUMENT LAYOUT NOTES:
%
% The next \usepackage line changes the layout for CV style section
% headings as marginal notes. It also sets up the paper size as either
% letter or A4. By default, letter was used. If A4 paper is desired,
% comment out the letterpaper lines and uncomment the a4paper lines.
%
% As you can see, the margin widths and section title widths can be
% easily adjusted.
%
% ALSO: Notice that the includefoot option can be commented OUT in order
% to put the PAGE NUMBER *IN* the bottom margin. This will make the
% effective text area larger.
%
% IF YOU WISH TO REMOVE THE ``of LASTPAGE'' next to each page number,
% see the note about the +LP and -LP lines below. Comment out the +LP
% and uncomment the -LP.
%
% IF YOU WISH TO REMOVE PAGE NUMBERS, be sure that the includefoot line
% is uncommented and ALSO uncomment the \pagestyle{empty} a few lines
% below.
%

%% Use these lines for letter-sized paper
\usepackage[paper=letterpaper,
            %includefoot, % Uncomment to put page number above margin
            marginparwidth=1.2in,     % Length of section titles
            marginparsep=.05in,       % Space between titles and text
            margin=0.65in,               % 1 inch margins
            includemp]{geometry}

%% Use these lines for A4-sized paper
%\usepackage[paper=a4paper,
%            %includefoot, % Uncomment to put page number above margin
%            marginparwidth=30.5mm,    % Length of section titles
%            marginparsep=1.5mm,       % Space between titles and text
%            margin=25mm,              % 25mm margins
%            includemp]{geometry}

%% More layout: Get rid of indenting throughout entire document
\setlength{\parindent}{0in}

%% This gives us fun enumeration environments. compactitem will be nice.
\usepackage{paralist}

%% Reference the last page in the page number
%
% NOTE: comment the +LP line and uncomment the -LP line to have page
%       numbers without the ``of ##'' last page reference)
%
% NOTE: uncomment the \pagestyle{empty} line to get rid of all page
%       numbers (make sure includefoot is commented out above)
%
\usepackage{fancyhdr,lastpage}
\pagestyle{fancy}
%\pagestyle{empty}      % Uncomment this to get rid of page numbers
\fancyhf{}\renewcommand{\headrulewidth}{0pt}
\fancyfootoffset{\marginparsep+\marginparwidth}
\newlength{\footpageshift}
\setlength{\footpageshift}
          {0.5\textwidth+0.5\marginparsep+0.5\marginparwidth-2in}
\lfoot{\hspace{\footpageshift}%
       \parbox{4in}{\, \hfill %
%                    \arabic{page} of \protect\pageref*{LastPage} % +LP
                    \arabic{page}                               % -LP
                    \hfill \,}}


% Finally, give us PDF bookmarks
\usepackage{color,hyperref}
\definecolor{lightblue}{RGB}{47, 25, 178}
\definecolor{superblue}{RGB}{0, 51, 127}
\definecolor{lightgray}{RGB}{108, 108, 108}
\hypersetup{colorlinks,breaklinks,
            linkcolor=lightblue,urlcolor=lightblue,
            anchorcolor=lightblue,citecolor=lightblue}

%%%%%%%%%%%%%%%%%%%%%%%% End Document Setup %%%%%%%%%%%%%%%%%%%%%%%%%%%%


%%%%%%%%%%%%%%%%%%%%%%%%%%% Helper Commands %%%%%%%%%%%%%%%%%%%%%%%%%%%%

% The title (name) with a horizontal rule under it
%
% Usage: \makeheading{name}
%
% Place at top of document. It should be the first thing.
\newcommand{\makeheading}[2]%
        {\hspace*{-\marginparsep minus \marginparwidth}%
         \begin{minipage}[b]{0.35\textwidth+\marginparwidth+\marginparsep}%
           {\bfseries #1}%\\[-0.15\baselineskip]%
         \end{minipage}%
         \begin{minipage}[b]{0.45\textwidth+\marginparwidth+\marginparsep}%
           \flushright{#2}%
         \end{minipage}
       }

%\newcommand{\makepara}[2]%
        %{\hspace*{-\marginparsep minus \marginparwidth}%
         %\begin{minipage}[t]{0.5\textwidth+\marginparwidth+\marginparsep}%
           %{#1}%
         %\end{minipage}
         %\begin{minipage}[t]{0.5\textwidth+\marginparwidth+\marginparsep}%
           %{#2}%
         %\end{minipage}}
\newcommand{\makepara}[1]%
        {\hspace*{-\marginparsep minus \marginparwidth}%
        \begin{minipage}[t]{\textwidth+\marginparwidth+\marginparsep}%
          {#1}
       \end{minipage}}

\newcommand{\myruler}
        {\hspace*{-\marginparsep minus \marginparwidth}%
        \begin{minipage}[t]{\textwidth+\marginparwidth+\marginparsep}%
          %\rule{\textwidth+\marginparwidth+\marginparsep}{1pt}%
          \rule{\columnwidth}{1pt}%
       \end{minipage}}
    
% The section headings
%
% Usage: \section{section name}
%
% Follow this section IMMEDIATELY with the first line of the section
% text. Do not put whitespace in between. That is, do this:
%
%       \section{My Information}
%       Here is my information.
%
% and NOT this:
%
%       \section{My Information}
%
%       Here is my information.
%
% Otherwise the top of the section header will not line up with the top
% of the section. Of course, using a single comment character (%) on
% empty lines allows for the function of the first example with the
% readability of the second example.
\renewcommand{\section}[2]%
        {\pagebreak[2]\vspace{1.3\baselineskip}%
         \phantomsection\addcontentsline{toc}{section}{#1}%
         \hspace{0in}%
         \marginpar{
           \raggedright \scshape \bf \textcolor{superblue}{\large #1}}#2}


% An itemize-style list with lots of space between items
\newenvironment{outerlist}[1][\enskip\textbullet]%
        {\begin{itemize}[#1]}{\end{itemize}%
         \vspace{-.6\baselineskip}}

% An environment IDENTICAL to outerlist that has better pre-list spacing
% when used as the first thing in a \section 
\newenvironment{lonelist}[1][\enskip\textbullet]%
        {\vspace{-\baselineskip}\begin{list}{#1}{%
        \setlength{\partopsep}{0pt}%
        \setlength{\topsep}{0pt}}}
        {\end{list}\vspace{-.6\baselineskip}}

% An itemize-style list with little space between items
\newenvironment{innerlist}[1][\enskip\textbullet]%
        {\begin{compactitem}[#1]}{\end{compactitem}}

% To add some paragraph space between lines.
% This also tells LaTeX to preferably break a page on one of these gaps
% if there is a needed pagebreak nearby.
\newcommand{\blankline}{\quad\pagebreak[2]}

\newcommand{\ccp}{C\nolinebreak\hspace{-.05em}\raisebox{.4ex}{\tiny\bf+}\nolinebreak\hspace{-.10em}\raisebox{.4ex}{\tiny\bf +}}

%%%%%%%%%%%%%%%%%%%%%%%% End Helper Commands %%%%%%%%%%%%%%%%%%%%%%%%%%%

%%%%%%%%%%%%%%%%%%%%%%%%% Begin CV Document %%%%%%%%%%%%%%%%%%%%%%%%%%%%

\begin{document}
%\makeheading{{\huge\textcolor{superblue}{Asif Imran}} \\[+0.15\baselineskip]
%\normalsize \textbf{Astrophysics -- Statistics}{\small Wisconsin
%IceCube Particle Astrophysics Center (Madison, WI)}}
%\makeheading{\textcolor{superblue}{Asif ~Imran}}
%
\makeheading{\huge \textcolor{superblue}{Asif Imran}\\[+0.15\baselineskip]
\normalsize \textbf{Astrophysics -- Statistics}}
{\small Wisconsin IceCube Particle Astrophysics Center\\
222 W. Washington Ave \#500, Madison, WI \\
Phone: (608)-890-0957 (office) \\
\href{mailto:aimran@icecube.wisc.edu}{\tt{aimran@icecube.wisc.edu}}} \\[-1.65\baselineskip]%

\myruler

\vspace{0.4\baselineskip}

\makepara {\textbf{\normalsize \textcolor{superblue}{\large Research
  Interests}}: Application of {\it cutting-edge} imaging techniques and statistical tools
  to analyze large quantities of data to study high-energy photon
  emissions from distant galaxies. 

%%\setdefaultleftmargin{0cm}{2cm}{}{}{}{}
%%\begin{itemize}
%%    \renewcommand{\labelitemi}{\textcolor{lightgray}{\scriptsize$\blacksquare$}}
%%  \item this is an item
%%  \item this is an item
%%  \item this is an item
%%\end {itemize}
}

%\section{Contact Information}
%%
%% NOTE: Mind where the & separators and \\ breaks are in the following
%%       table.
%%
%% ALSO: \rcollength is the width of the right column of the table 
%%       (adjust it to your liking; default is 1.85in).
%%
%\newlength{\rcollength}\setlength{\rcollength}{1.85in}%
%%
%\begin{tabular}[t]{@{}p{\textwidth-\rcollength}p{\rcollength}}

%\href{http://wipac.wisc.edu/}%
     %{Wisconsin IeCube Particle Astrophysics Center} & \\
%222 W. Washington Avenue & Office: (608) 890-0446 \\
%Suite 500          & Fax:    ~~~(608) 262-2309 \\
%Madison, WI 53703  &\href{mailto:aimran@icecube.wisc.edu}{\tt{aimran@icecube.wisc.edu}}\\
%\end{tabular}

%\section{Research Interests}
%
%Observational gamma-ray astronomy, active galactic nuclei

\section{Education}
%
\textbf{Iowa State University}, 
Ames, IA 
\begin{innerlist}
\item[] Ph.D, Astrophysics
  \hfill \textcolor{lightgray}{June 2010}
\end{innerlist}
\blankline

{\textbf{Grinnell College}, 
Grinnell, IA 
\begin{innerlist}
\item[] B.A., Physics \emph{With Honors}
  \hfill \textcolor{lightgray}{May 2003}
\end{innerlist}

\section{Skills} 
\textbf{Hardware:} \href{http://www.hawc-observatory.org}{HAWC Observatory}

\begin{itemize}
    \renewcommand{\labelitemi}{\textcolor{lightgray}{\scriptsize$\blacksquare$}}

    \item  Designed and built real-time, VMEbus-based data acquisition system
      capable of handling an unprecedented 500 MBytes/second.  The novel design
      forgoes traditional hardware trigger in favor of purely software-based
      triggers.

  \end{itemize}



\textbf{Software}
\begin{itemize}
    \renewcommand{\labelitemi}{\textcolor{lightgray}{\scriptsize$\blacksquare$}}
  \item Currently developing a \ccp-based package with a tightly integrated
      SQLite-DB backend for the real-time monitoring of gamma-ray emission to be
      used by the HAWC collaboration.
  \item Developed C-based libraries for synchronized readout
    of an array of single board computers with a net throughput rate of 500
    MBytes/sec.

  \item Developed Monte Carlo simulation package for the VERITAS
      collaboration. Developed software analysis tools for the HAWC
      collaboration to measure sensitivity of the detector.


  \item Languages: Python, C, \ccp, database query languages (SQLite and MySQL).

  \item Tools: ROOT, IPython, NumPy, SciPy, Matplotlib, Boost,
    Pandas, Pyfits, SQLAlchemy, \LaTeX, SVN, Git, bash \& regexp.


\end{itemize}

\section{Research Experience}
\textbf{Wisconsin Icecube Particle Astrophysics Center} - Madison, WI
\begin{innerlist}
\item[] Postdoctoral Research Associate
  \hfill \textcolor{lightgray}{2013 -- present}
\item[] \textit{Supervisor: Stephan Westerhoff}
\item[] Develop tools for Fast, Real-Time Monitoring of Gamma Ray Emission with HAWC Observatory.
\end{innerlist}


\blankline{}

\textbf{Los Alamos National Laboratory} - Los Alamos, NM
\begin{innerlist}
\item[] Postdoctoral Research Associate
  \hfill \textcolor{lightgray}{2010 -- 2013}
\item[] \textit{Supervisor: Brenda Dingus}
\item[] Design and Develop VME-based Data Acquisition System for the HAWC Experiment.
\item[] Study of Extragalactic Gamma-Ray Sources with data from HAWC and the
  Fermi-Space Telescope.
\end{innerlist}


\blankline{}

\textbf{Iowa State University} -Ames, IA
\begin{innerlist}
\item[] Graduate Student Researcher
  \hfill \textcolor{lightgray}{2005 -- 2010}
\item[] \textit{Advisor: Frank Krennrich}
\item[] Analyzed Variable Gamma-Ray Emission from Active Galactic Nuclei.
\item[] Assembled \& Tested camera electronics for the VERITAS telescopes.
\end{innerlist}


\blankline{}

\textbf{Grinnell College} - Grinnell, IA
\begin{innerlist}
\item[] Undergraduate Mentored Advanced Project
  \hfill \textcolor{lightgray}{2002}
\item[] \textit{Advisor: Charlie Duke}
\item[] Simulation of Cherenkov Photons from Cosmic-Ray Cascade in the Earth's Atmosphere.
\end{innerlist}


\section{Grants} 
%
\textbf{Co-Investigator}, \textit{A Search for Unique Signatures from Extragalactic Background Light (EBL) Absorption Effects in TeV Blazar Spectra}, Fermi Guest Investigator Cycle 2 Grant 2009-2010 (PI: F. Krennrich).


\section{Awards and Honors} 
%
\textbf{Iowa State University}
\begin{innerlist}

\item[] Graduate teaching excellence award% 
  \hfill \textcolor{lightgray}{2005}
\item[] Teaching assistant of the year, Dept. of Physics and Astronomy% 
  \hfill \textcolor{lightgray}{2004}
\item[] Hardware scholarship, Dept. of Physics and Astronomy%
  \hfill \textcolor{lightgray}{2003 - 2005}

\end{innerlist}

\blankline

\textbf{Grinnell College}
\begin{innerlist}
\item[] International Merit Scholarship%
  \hfill \textcolor{lightgray}{1999 -- 2003}
\end{innerlist}


\section{Teaching Experience}
\textbf{Undergraduate Student Mentor}
\begin{innerlist}
\item[] Stephen Sturdevant (University of Wisconsin- Madison)
  \hfill \textcolor{lightgray}{2013 -- Present}
\end{innerlist}
\blankline{}

\textbf{Instructor}
\begin{innerlist}
\item[] WIPAC High School Internship Program
  \hfill \textcolor{lightgray}{Fall 2013}
\item[] Co-taught high school students about basic electronic circuits and
  building data acquisition system with arduino boards.
\end{innerlist}

\blankline{}

\textbf{Graduate Student Mentor}
\begin{innerlist}
\item[] Peter Karn (UC Irvine)
  \hfill \textcolor{lightgray}{2011 -- 2013}
\end{innerlist}

\blankline{}

\textbf{Teaching Assistant}
\begin{innerlist}
\item[] Department of Physics and Astronomy
  \hfill \textcolor{lightgray}{2003 -- 2005}
\item[] {Iowa State University, Ames, Iowa}
\item[] Performed TA duties and conducted help sessions for Astronomy 120/150 
       (introductory astronomy courses), Astronomy 346 (intermediate astrophysics course) 
        and Physics 221/222 (intermediate physics courses).
\end{innerlist}

\blankline{}

\textbf{Teaching Assistant/Tutor}
\begin{innerlist}
\item[] Department of Physics
  \hfill \textcolor{lightgray}{2001 -- 2003}
\item[] {Grinnell College, Grinnell, Iowa}
\item[] Teacher's support for introductory physics workshop. Helped students identify problem
        areas and provided timely feedbacks.
\end{innerlist}

\blankline{}

\textbf{Mathematics Lab Tutor}
\begin{innerlist}
\item[] {Grinnell College, Grinnell, Iowa}
  \hfill \textcolor{lightgray}{2001 -- 2003}
\item[] Provided structured mentoring to students. Conducted one-on-one help
        sessions aimed at supplementing class lecturers.
       
\end{innerlist}


\section{Conference and Workshop}
APS 4-Corners Section Meeting (2012), Socorro, NM. \textit{(Invited)} \\
The 32$^{nd}$ International Cosmic Ray Conference (2011), Beijing, China.\\
APS April Meeting (2011), Anaheim, CA. \\
INPAC Meeting (2011), Asilomar, CA. \textit{(Invited)} \\
The 31$^{st}$ International Cosmic Ray Conference (2009), Lodz, Poland. \\

\newpage

\section{Selected Peer Reviewed Publications }
%
\textit{Sensitivity of the high altitude water Cherenkov detector to sources of
multi-TeV gamma rays}, Abeysekara, A. U., et al. for the HAWC Collaboration,
\underline{Astroparticle Physics}, \textbf{50} (2013), 26A

\blankline{}

\textit{Constraints on Cosmic Rays, Magnetic Fields, and Dark Matter from Gamma-Ray Observations of the Coma Cluster of Galaxies with VERITAS and Fermi}, Arlen, T., et al. for the VERITAS Collaboration, \underline{Astrophysical Journal}, \textbf{757} (2012), 123.

\blankline{}

\textit{On the Sensitivity of the HAWC Observatory}, Abeysekara, A. U., for the HAWC \\Collaboration, \underline{Astroparticle Physics}, \textbf{35} (2012), 641.

\blankline{}

\textit{Dectection of Pulsed Gamma Rays Above 100 GeV from the Crab Pulsar}, Aliu, E., for the VERITAS Collaboration, \underline{Science}, \textbf{334} (2011), 69.

\blankline{}

\textit{Discovery of Very High Energy $\gamma$-ray from the SNR G54.1+0.3}, Acciari, V., et al. for the VERITAS Collaboration, \underline{Astrophysical Journal Letters}, \textbf{719} (2010), 69.

\blankline{}

\textit{VERITAS discovery of variability in the very high energy $\gamma$-ray emission of 1ES 1218+304}, Acciari, V., et al. for the VERITAS Collaboration, \underline{Astrophysical Journal Letters}, \textbf{709L} (2010), 163.

\blankline{}

\textit{A connection between star formation activity and cosmic rays in the starburst galaxy M~82}, Acciari, V., et al. for the VERITAS Collaboration, \underline{Nature}, \textbf{462} (2009), 770.

\blankline{}


\textit{VERITAS upper limit on the very high energy emission from the radio galaxy NGC 1275}, Acciari, V., et al. for the VERITAS Collaboration, \underline{Astrophysical Journal Letters}, \textbf{706L} (2009), 275. 

\blankline{}

\textit{Radio imaging of the very-high-energy $\gamma$-ray emission region in the central engine of a radio galaxy}, Acciari, V., et al. for the VERITAS Collaboration, \underline{Science}, \textbf{325} (2009), 444.


\blankline{}

\textit{Observation of extended very high energy emission from the supernova remnant IC 443 with VERITAS}, Acciari, V., et al. for the VERITAS Collaboration, \underline{Astrophysical Journal}, \textbf{698L} (2009), 133.

\blankline{}


\textit{Evidence for long-term gamma-ray and x-ray variability from the unidentified TeV source HESS J0632+057}, Acciari, V., et al. for the VERITAS Collaboration, \\\underline{Astrophysical Journal Letters}, \textbf{698L} (2009), 94.

\blankline{}


\textit{VERITAS observations of the BL Lac object 1ES 1218+304}, Acciari, V., et al. for the VERITAS Collaboration, \underline{Astrophysical Journal}, \textbf{695} (2009), 1370.

\blankline{}

\textit{VERITAS observations of a very high energy gamma-ray flare from the blazar 3C 66A}, Acciari, V., et al. for the VERITAS Collaboration, \underline{Astrophysical Journal Letters}, \textbf{693L} (2009), 104.

\blankline{}

\textit{The June 2008 flare of Markarian 421 from optical to TeV energies}, Acciari, V., et al. for the VERITAS Collaboration, \underline{Astrophysical Journal Letters}, \textbf{691L} (2009), 13.

\blankline{}


\textit{Discovery of very high energy gamma-ray radiation from the BL Lac 1ES 0806+524}, Acciari, V., et al. for the VERITAS Collaboration, \underline{Astrophysical Journal}, \textbf{690L} (2009), 126.

 \blankline{}

 \textit{Constraints on energy spectra of blazars based on recent EBL limits from galaxy counts}, Krennrich, F., Dwek, E., \& Imran, A., \underline{Astrophysical Journal Letters}, \textbf{689L} (2008), 93.

 %\blankline{}

 %\textit{VERITAS discovery of $>200$ GeV gamma-ray emission from the intermediate-frequency-peaked BL Lacertae object W Comae}, Acciari, V., et al. for the VERITAS Collaboration, \underline{Astrophysical Journal Letters}, \textbf{684L} (2008), 73.

\blankline{}

 \textit{Observation of gamma-ray emission from the galaxy M87 above 250 GeV with VERITAS}, Acciari, V., et al. for the VERITAS Collaboration, \underline{Astrophysical Journal}, \textbf{679} (2008), 1427.

 \blankline{}

% \textit{Observations of the unidentified TeV $\gamma$-ray source TeV J2032+4130 with the Whipple observatory 10 m telescope}, Konopelko, A., et al. for the Whipple Collaboration, \underline{Astrophysical Journal}, \textbf{658} (2007), 1062.

%% \blankline{}

%% \textit{The first VERITAS telescope}, Holder, J., et al. for the VERITAS Collaboration,  \underline{Astroparticle Physics}, \textbf{25} (2006), 391.

%% \blankline{}

%% %\newpage
%%\section{Non-peer Reviewed Publications \\
%%and Posters}
%%%% %
%%\textit{Study of HAWC Sensitivity to Active Galactic Nuclei}, Imran, A. for the HAWC Collaboration, in Proc. to the 32$^{nd}$ International Cosmic Ray Conference, Beijing, China (2011)
%%
%%\blankline{}
%%
%%\textit{VERITAS discovery of variability in the very high energy $\gamma$-ray emission of 1ES 1218+304},  Imran, A. for the VERITAS Collaboration, in Proc. to the 31$^{st}$ International Cosmic Ray Conference, Lodz, Poland (2009).
%%
%%\blankline{}
%%
%%\textit{Detecting a unique EBL signature with TeV gamma rays}, Imran, A., Krennrich, F., in Proc. to the 30$^{th}$ International Cosmic Ray Conference, Merida, Mexico (2007).
%%
%%\blankline{}
%%
%%\textit{Searching for EBL-absorption features in the TeV blazar spectrum}, Imran, A. 17th Ann. Astrophys. Conf., College Park, MD (2006)
%%
%%\newpage

%\blankline{}


%% \blankline

%% Instrumentation and Control: 
%%         \href{http://www.ni.com/}{LabVIEW} 
        

%\blankline

%Tools used: C, C++, ROOT, Python, Perl, UNIX shell scripting, MySQL
%%          CVS, SVN. 

%\blankline

%%Applications: \LaTeX{}, B\textsc{ib}\TeX{}, Microsoft Office,
%%         and other common productivity packages for Windows, OS X, and
%%         Linux platforms

%% \blankline

%%Operating Systems: Linux, Apple OS X, BSD


% \newpage{}
% \section{List of References}
% %
% \begin{innerlist}
% \item[] \textbf{Frank Krennrich}
% \item[] Professor of Physics \& Astronomy
% \item[] Iowa State University
% \item[] 12 Physics Hall, Ames, IA 50011
% \item[] \href{mailto:krennich@iastate.edu}{\tt{krennich@iastate.edu}}
% \end{innerlist}


% \blankline


% \begin{innerlist}
% \item[] \textbf{Trevor Weekes}
% \item[] Senior Astrophysicist
% \item[] Harvard-Smithsonian Center for Astrophysics
% \item[] Whipple Observatory, 670 Mt. Hopkins Road,
% \item[] P.O. Box 6369, Amado, AZ 85645-0097
% \item[] \href{mailto:weekes@veritas.sao.arizona.edu}{\tt{weekes@veritas.sao.arizona.edu}}
% \end{innerlist}

% \blankline

% \begin{innerlist}
% \item[] \textbf{Charles Duke}
% \item[] S. S. Williston Professor of Physics
% \item[] Grinnell College
% \item[] 1333 Park Street, Grinnell, IA 50112
% \item[] \href{mailto:duke@grinnell.edu}{\tt{duke@grinnell.edu}}
% \end{innerlist}


% \blankline

% \begin{innerlist}
% \item[] \textbf{David Carter-Lewis}
% \item[] Professor of Physics \& Astronomy
% \item[] Iowa State University
% \item[] 12 Physics Hall, Ames, IA 50011
% \item[] \href{mailto:dalewis@iastate.edu}{\tt{dalewis@iastate.edu}}
% \end{innerlist}


% \blankline

\end{document}

%%%%%%%%%%%%%%%%%%%%%%%%%% End CV Document %%%%%%%%%%%%%%%%%%%%%%%%%%%%%
% \section{Research Appointments}
% \href{http://www.physastro.iastate.edu}{\textbf{Iowa State University}}, 
% Ames, Iowa
% \begin{outerlist}
% \item[] \textit{Graduate Student Researcher}%
%         \hfill \textbf{August 2005 -- present}
% \begin{innerlist}
% \item[]Advisor: \href{http://cherenkov.physics.iastate.edu/}{Frank ~Krennrich}\\
%       \textit{}
%       \begin{innerlist}
%         \item something
%         \item something 
%         \end{innerlist}
% \end{innerlist}
% \end{outerlist}

% \blankline{}

% \href{http://www.grinnell.edu}{\textbf{Grinnell College}}, 
% Grinnell, Iowa
% \begin{outerlist}
% \item[] \textit{Undergraduate Mentored Advanced Project (MAP)}% 
%         \hfill \textbf{May 2002 -- August 2002}
% \begin{innerlist}
% \item[]Advisor: \href{http://cherenkov.physics.iastate.edu/}{Charlie ~Duke}\\
%       \textit{Thesis topic here}
%       \begin{innerlist}
%         \item something
%         \item something 
%         \end{innerlist}
% \end{innerlist}


% \item[] \textit{Undergraduate Research Assistant}%
%         \hfill \textbf{May 2001 -- August 2001}
% \begin{innerlist}
% \item[]Advisor:~{Paul ~Bunson}\\
%       \textit{Thesis topic here}
%       \begin{innerlist}
%         \item something
%         \item something 
%         \end{innerlist}
% \end{innerlist}

% \end{outerlist}
